% Created 2023-10-11 mié 12:10
% Intended LaTeX compiler: lualatex
\documentclass[11pt]{article}
\usepackage{graphicx}
\usepackage{longtable}
\usepackage{wrapfig}
\usepackage{rotating}
\usepackage[normalem]{ulem}
\usepackage{amsmath}
\usepackage{amssymb}
\usepackage{capt-of}
\usepackage{hyperref}
\usepackage{polyglossia}
\setmainlanguage{spanish}
\usepackage{csquotes}
\usepackage{emoji}
\usepackage[backend=biber, style=alphabetic, backref=true]{biblatex}
\addbibresource{tangled/biblio.bib}
\author{Christian Gimenez}
\date{16 ago 2023}
\title{Programa 29}
\hypersetup{
 pdfauthor={Christian Gimenez},
 pdftitle={Programa 29},
 pdfkeywords={},
 pdfsubject={},
 pdfcreator={Emacs 30.0.50 (Org mode 9.6.2)}, 
 pdflang={Spanish}}
\begin{document}

\maketitle
\tableofcontents

Emisión al aire: \textit{<2023-10-11 mié 13:00>}

\begin{table}[tbp]
\caption{Guión técnico para el operador.}
\centering
\begin{tabular}{llrr}
Guión técnico & Detalle & Duración & Finaliza\\[0pt]
\hline
Comienzo & Cortinilla: \texttt{Cortinillas/Apertura Ctrl Z.mp3} & 00:00:10 & 13:00:10\\[0pt]
Presentación & Presentación del programa & 00:02:00 & 13:02:10\\[0pt]
\hline
Noticias &  & 00:05:00 & 13:07:10\\[0pt]
\hline
\emoji{musical-note} Música & \emoji{bomb} Bomba: ¡Todo tuyo! & 00:03:00 & 13:10:10\\[0pt]
\hline
Bloque 1 & Hornero en las escuelas & 00:15:00 & 13:25:10\\[0pt]
 & \emoji{telephone} Llamar a Fede Ceccotti &  & \\[0pt]
\hline
\emoji{light-bulb} Tips & \emoji{bomb} Bomba: Fondo Rainbow Tylenol & 00:02:00 & 13:27:10\\[0pt]
\hline
\emoji{pause-button} Tanda & ----- Tanda Publicitaria ----- & 00:05:00 & 13:32:10\\[0pt]
\hline
Noticia deportiva & \emoji{telephone} Llamar o hablar con Ian & 00:05:00 & 13:37:10\\[0pt]
\hline
Bloque 2 & Observatorio Electoral en las elecciones & 00:15:00 & 13:52:10\\[0pt]
 & \emoji{telephone} Llamar a Soledad Anselmi &  & \\[0pt]
 & \emoji{play-button} Reproducir Audio de spot del observatorio &  & \\[0pt]
\hline
\emoji{light-bulb} Tips & \emoji{bomb} Bomba: Fondo Rainbow Tylenol & 00:02:00 & 13:54:10\\[0pt]
\hline
Cierre & Cortinilla/separador: \texttt{Cortinillas/Cierre Ctrl Z.mp3} & 00:01:00 & 13:55:10\\[0pt]
\hline
\hline
\textbf{Total} &  & 00:55:10 & \\[0pt]
\end{tabular}
\end{table}

\section{Presentación del programa}
\label{sec:orgfcb2075}
\begin{itemize}
\item Bienvenida a Control Z, el espacio de la Facultad de Informática.
\item Miércoles a las 13:00 de la tarde
\item Presentarse
\item Resumen de qué se trata Control Z: ciudadanía digital, cultura libre, estudiá computación, informática diversa, control N
\item Canales de comunicación: Instagram: \href{https://www.instagram.com/ctrlz\_fai/}{ctrlz\_fai} | Web: \url{https://controlz.fi.uncoma.edu.ar/} | Correo: controlz@fi.uncoma.edu.ar
\item Resumen del programa de hoy: ¿qué hay hoy?
\end{itemize}

\section{Noticias}
\label{sec:org4d1c7a0}

\subsection{Internacionales}
\label{sec:org9f29a25}

\begin{itemize}
\item Segundo martes de Octubre se conoce como el día de Ada Lovelace. Honrando las contribuciones de las mujeres en las ciencias, la tecnología, la ingeniería y las matemáticas (STEM).
\url{https://www.nationalgeographicla.com/ciencia/2023/10/fue-la-primera-programadora-de-la-historia-y-predijo-la-existencia-de-la-inteligencia-artificial}
\item Las IA empeoran aún más las respuestas rápidas de Google: \href{https://arstechnica.com/information-technology/2023/09/can-you-melt-eggs-quoras-ai-says-yes-and-google-is-sharing-the-result/}{Can you melt eggs? Quora's AI says "yes", and Google is sharing the result - ARS Technica}.
\item "Robotaxi parks on woman’s leg after running her over".

\url{https://www.telegraph.co.uk/world-news/2023/10/03/san-francisco-cruise-driverless-car-woman-hit-and-run-crash/}
\end{itemize}


\subsection{Nacionales}
\label{sec:orgbf55ebc}
\begin{itemize}
\item Clementina XXI comenzó a funcionar. Una supercomputadora adquirida en diciembre del 2022 se puso en funcionamiento el 27 de septiembre. Es una de las 100 supercomputadoras más poderosas del mundo. \href{https://www.pagina12.com.ar/592307-clementina-xxi-la-supercomputadora-argentina-ya-comenzo-a-fu}{Fuente}.
\end{itemize}

\subsection{FaiWeb}
\label{sec:org219428a}
\begin{itemize}
\item Carrera de la universidad: "UNCo Activa". Amplía nuestro columnista en deportes @Ian.
\item Extende inicio de Curso de Posgrado "Big Data: Procesos, Componentes y Herramientas". a Dra. Agustina Buccella. Inicia: \textbf{Jueves 19} de octubre de 2023. Comunicarse por mail a posgradofai@fi.uncoma.edu.ar. \href{https://www.fi.uncoma.edu.ar/index.php/investigacion-y-postgrado/cursos/curso-de-posgrado-big-data-procesos-componentes-y-herramientas/}{Fuente.}
\item Cristian Vincenzini: Modelos de generación de comentarios de código basados en transformers. El 3 de octubre de 2023, el estudiante Cristian Vincenzini aprobó su tesis de Licenciatura en Ciencias de la Computación. ¡Felicitaciones Licenciado!
\item Pasayo:
\begin{itemize}
\item 75 estudiantes en el espectro en total de los cuales: 
\begin{itemize}
\item 55 corresponden a facilitaciones de la Escuela PASAYO
\item 20 corresponden a estudiantes de docentes o terapeutas haciendo el trayecto de formación docente.
\end{itemize}
\item 03 familias están iniciando el nivel TANGIBLE
\item 12 familias están avanzandas en el nivel TANGIBLE
\item 37 familias han completado el nivel TANGIBLE
\item 12 familias están iniciando el nivel BLOQUES
\item 26 familias están avanzandas en el nivel
\end{itemize}
\item Atención estudiantes de Licenciatura en Sistemas de Información: Cambios en los contenidos mínimos de algunas materias. ¡Ver en FaiWeb! \href{https://www.fi.uncoma.edu.ar/index.php/novedades/importante-atencion-estudiantes-de-licenciatura-en-sistemas-de-informacion/}{Fuente}.
\end{itemize}

\section{Bloque 1: Hornero en las escuelas}
\label{sec:org4f91972}
Presentar: Federico Ceccotti

Realizaron un torneo utilizando hornero entre la ESRN17 (Cipolletti) y el CET30 (Cipolletti).

Contexto:
\begin{itemize}
\item ¿Cómo contactaste a la universidad? ¿qué actividades han realizado previamente?
\item ¿Qué es Hornero? ¿Para que lo han utilizado?
\end{itemize}

Actividades actuales:
\begin{itemize}
\item Y ahora, ¿qué hicieron con Hornero? ¿adaptaron el software? ¿qué actividades han hecho?
\item ¿quiénes instalaron y/o adaptaron el software?
\item ¿qué cursos participaron de los torneos? ¿les gustó participar a lxs estudiantes?
\end{itemize}

A futuro:
\begin{itemize}
\item ¿van a hacer más torneos? ¿piensan mejorar Hornero?
\item ¿les sirvió el software que produjo la universidad? ¿hubiera sido posible si no fuese una universidad pública?
\item ¿tienen pensado participar del Programate?
\item ¿piensan hacer torneos con otras escuelas?
\end{itemize}

Mencionar invitadxs y despedir.

\section{Tips de Leo}
\label{sec:org76e5dc3}
\begin{itemize}
\item ¿Una vida sin contraseñas? Los desafíos de la ciberseguridad. \href{https://tn.com.ar/tecno/internet/2023/09/24/una-vida-sin-contrasenas-los-desafios-de-la-ciberseguridad/}{Fuente.}

Nos la dejaste picando la semana pasada\ldots{} ¡ahora contá! \emoji{laughing}
\end{itemize}

\section{\emoji{pause-button} Tanda}
\label{sec:org593e9fe}
\section{Noticia deportiva}
\label{sec:orgd9918fe}
\begin{itemize}
\item Carrera de la universidad: "UNCo Activa".
\item Preinscripciones en \href{https://uncoactiva.fi.uncoma.edu.ar}{uncoactiva.fi.uncoma.edu.ar}.
\item Sábado 28 de octubre 18:00, polideportivo "Beto Monteros", Neuquén.
\item ¿Sorteo para ganar dos entradas? \(\to\) ¿ya está listo?
\end{itemize}
\section{Bloque 2: Observatorio Electoral en las elecciones del 22 de octubre}
\label{sec:org42e1ace}
\emoji{play-button} Repetir Convocatoria: Reproducir spot de la radio.

Presentar: Soledad Anselmi, participó como observadora en varias elecciones previas.

\begin{itemize}
\item ¿Qué tareas hace un observador?
\item ¿Por qué observar las elecciones?
\item ¿Qué hacen con las observaciones?
\item ¿Qué observaciones han hecho?
\begin{itemize}
\item ¿Dónde puedo obtener información de observaciones anteriores?
\end{itemize}
\item ¿Cómo inscribirse?
\begin{itemize}
\item Para personas mayores de 18 años
\item Observación el día 22 de octubre
\item Inscribirse por observatorioelectoral.uncoma.edu.ar o enviar mensaje al 294 459-4321
\end{itemize}
\end{itemize}

Mencionar invitadxs y despedir.

\section{Tips de Leo}
\label{sec:org648bd5d}
\begin{itemize}
\item Clickjacking, la técnica de ciberestafa que es tendencia: de qué se trata y cómo detectarla. \href{https://tn.com.ar/tecno/novedades/2023/09/22/clickjacking-la-tecnica-de-ciberestafa-que-te-hace-hacer-en-internet-cosas-que-no-queres/}{Fuente.}
\end{itemize}
\end{document}